%  LaTeX support: latex@mdpi.com
%  In case you need support, please attach all files that are necessary for compiling as well as the log file, and specify the details of your LaTeX setup (which operating system and LaTeX version / tools you are using).

%=================================================================
\documentclass[journal,article,atmosphere,submit,moreauthors,pdftex]{Definitions/mdpi}

% If you would like to post an early version of this manuscript as a preprint, you may use preprint as the journal and change 'submit' to 'accept'. The document class line would be, e.g., \documentclass[preprints,article,accept,moreauthors,pdftex]{mdpi}. This is especially recommended for submission to arXiv, where line numbers should be removed before posting. For preprints.org, the editorial staff will make this change immediately prior to posting.

%--------------------
% Class Options:
%--------------------
%----------
% journal
%----------
% Choose between the following MDPI journals:
% acoustics, actuators, addictions, admsci, aerospace, agriculture, agriengineering, agronomy, ai, algorithms, animals, antibiotics, antibodies, antioxidants, applmech, applsci, arts, asc, asi, atmosphere, atoms, axioms, batteries, bdcc, behavsci , beverages, bioengineering, biology, biomedicines, biomimetics, biomolecules, biosensors, brainsci , buildings, cancers, carbon , catalysts, cells, ceramics, challenges, chemengineering, chemistry, chemosensors, children, civileng, cleantechnol, climate, clockssleep, cmd, coatings, colloids, computation, computers, condensedmatter, cosmetics, cryptography, crystals, dairy, data, dentistry, designs , diagnostics, diseases, diversity, drones, econometrics, economies, education, ejbc, ejihpe, electrochem, electronics, endocrines, energies, entropy, environments, epigenomes, est, fermentation, fibers, fire, fishes, fluids, foods, forecasting, forests, fractalfract, futureinternet, futurephys, galaxies, games, gastrointestdisord, gels, genealogy, genes, geohazards, geosciences, geriatrics, hazardousmatters, healthcare, hearts, heritage, highthroughput, horticulturae, humanities, hydrology, ijerph, ijfs, ijgi, ijms, ijtpp, informatics, information, infrastructures, inorganics, insects, instruments, inventions, iot, j, jcdd, jce, jcm, jcp, jcs, jdb, jfb, jfmk, jimaging, jintelligence, jlpea, jmmp, jmse, jne, jnt, jof, joitmc, jpm, jrfm, jsan, land, languages, laws, life, literature, logistics, lubricants, machines, magnetochemistry, make, marinedrugs, materials, mathematics, mca, medicina, medicines, medsci, membranes, metabolites, metals, microarrays, micromachines, microorganisms, minerals, modelling, molbank, molecules, mps, mti, nanomaterials, ncrna, ijns, neurosci, neuroglia, nitrogen, notspecified, nutrients, oceans, ohbm, optics, particles, pathogens, pharmaceuticals, pharmaceutics, pharmacy, philosophies, photonics, physics, plants, plasma, pollutants, polymers, polysaccharides, preprints , proceedings, processes, prosthesis, proteomes, psych, publications, quantumrep, quaternary, qubs, reactions, recycling, religions, remotesensing, reprodmed, reports, resources, risks, robotics, safety, sci, scipharm, sensors, separations, sexes, signals, sinusitis, smartcities, sna, societies, socsci, soilsystems, sports, standards, stats, surfaces, surgeries, sustainability, sustainableworld, symmetry, systems, technologies, telecom, test, tourismhosp, toxics, toxins, transplantology, tropicalmed, universe, urbansci, vaccines, vehicles, vetsci, vibration, viruses, vision, water, wem, wevj

%---------
% article
%---------
% The default type of manuscript is "article", but can be replaced by:
% abstract, addendum, article, benchmark, book, bookreview, briefreport, casereport, changes, comment, commentary, communication, conceptpaper, conferenceproceedings, correction, conferencereport, expressionofconcern, extendedabstract, meetingreport, creative, datadescriptor, discussion, editorial, essay, erratum, hypothesis, interestingimages, letter, meetingreport, newbookreceived, obituary, opinion, projectreport, reply, retraction, review, perspective, protocol, shortnote, supfile, technicalnote, viewpoint
% supfile = supplementary materials

%----------
% submit
%----------
% The class option "submit" will be changed to "accept" by the Editorial Office when the paper is accepted. This will only make changes to the frontpage (e.g., the logo of the journal will get visible), the headings, and the copyright information. Also, line numbering will be removed. Journal info and pagination for accepted papers will also be assigned by the Editorial Office.

%------------------
% moreauthors
%------------------
% If there is only one author the class option oneauthor should be used. Otherwise use the class option moreauthors.

%---------
% pdftex
%---------
% The option pdftex is for use with pdfLaTeX. If eps figures are used, remove the option pdftex and use LaTeX and dvi2pdf.

%=================================================================
\firstpage{1}
\makeatletter
\setcounter{page}{\@firstpage}
\makeatother
\pubvolume{xx}
\issuenum{1}
\articlenumber{5}
\pubyear{2020}
\copyrightyear{2020}
%\externaleditor{Academic Editor: name}
\history{Received: date; Accepted: date; Published: date}
%\updates{yes} % If there is an update available, un-comment this line

%% MDPI internal command: uncomment if new journal that already uses continuous page numbers
%\continuouspages{yes}

%------------------------------------------------------------------
% The following line should be uncommented if the LaTeX file is uploaded to arXiv.org
%\pdfoutput=1

%=================================================================
% Add packages and commands here. The following packages are loaded in our class file: fontenc, inputenc, calc, indentfirst, fancyhdr, graphicx,epstopdf, lastpage, ifthen, lineno, float, amsmath, setspace, enumitem, mathpazo, booktabs, titlesec, etoolbox, tabto, xcolor, soul, multirow, microtype, tikz, totcount, amsthm, hyphenat, natbib, hyperref, footmisc, url, geometry, newfloat, caption

%=================================================================
%% Please use the following mathematics environments: Theorem, Lemma, Corollary, Proposition, Characterization, Property, Problem, Example, ExamplesandDefinitions, Hypothesis, Remark, Definition, Notation, Assumption
%% For proofs, please use the proof environment (the amsthm package is loaded by the MDPI class).

%=================================================================
% Full title of the paper (Capitalized)
\Title{Application of a Cut-Cell Immersed Boundary Method for Wildfire Simulation over Complex Terrain}

% Author Orchid ID: enter ID or remove command
\newcommand{\orcidauthorA}{0000-0000-000-000X} % Add \orcidA{} behind the author's name
%\newcommand{\orcidauthorB}{0000-0000-000-000X} % Add \orcidB{} behind the author's name

% Authors, for the paper (add full first names)
\Author{Marcos Vanella $^1$, Kevin McGrattan $^1$, Randall McDermott $^1$, Glenn Forney $^1$, and William Mell $^2$}

% Authors, for metadata in PDF
\AuthorNames{Marcos Vanella, Kevin McGrattan, Randall McDermott, Glenn Forney, and William Mell}

% Affiliations / Addresses (Add [1] after \address if there is only one affiliation.)
\address{%
$^{1}$ \quad National Institute of Standards and Technology, Gaithersburg, Maryland, USA \\
$^{2}$ \quad U.S. Forest Service, Seattle, Washington, USA}

% Contact information of the corresponding author
\corres{Correspondence: marcos.vanella@nist.gov}


\abstract{A method for the large-eddy simulation (LES) of wildfire spread over complex terrain is presented. In this scheme, a cut-cell immersed boundary method (CC-IBM) is used to render the complex terrain, defined by a tessellation, on a rectilinear Cartesian grid. Discretization of scalar transport equations for chemical species is done via a finite volume scheme on cut-cells defined by the intersection of the terrain geometry and the Cartesian cells. Momentum transport and heat transfer close to the immersed terrain are handled using dynamic wall models and a direct forcing immersed boundary method. Further, atmospheric wind conditions are specified using a mean forcing concept. Fire spread is modeled using one of three basic approaches: (1) representing the vegetation as a collection of Lagrangian particles, (2) representing the vegetation as a semi-porous boundary, and (3) representing the fire spread using a level set method in which the fire spreads as a function of terrain slope, vegetation type, and wind speed. Several test and validation cases are reported to demonstrate the capabilities of this novel wildfire simulation methodology.}

\keyword{complex terrain; fire spread; immersed boundary method; level sets}


\begin{document}


\section{Introduction}

% Motivation.
In the last decades, wild land and wild land-urban interface (WUI) fires have received substantial attention due to their destructive nature and extensive cost in lives and property~\cite{thomas_2017,mcdermott_2019,boghani_2019,richards_2020}. Over the years, modeling and simulation activities of such phenomena have been tailored to the needs of planners, scientists and first responders, and by the available computing capabilities.
% Wildfire modeling in the literature.
A large body of literature can be found on the subject. See for example reviews~\cite{Papadopoulos_2011,Bakhshaii_2019,mcdermott_2019} on classical and new models with associated software. Due to inherent complex nature of the problem, most models of wildfires have historically relied on simplified field-tested rules and correlations. 

Among these, the fire spread model by Rothermel~\cite{Rothermel:1972} assumes quasi-steady state ground fire conditions and takes as inputs fuel properties, wind, terrain slope and moisture that can be measured in situ. 
Fuel models have been cataloged~\cite{Anderson:1982}, and their input parameters defined in a manner the user can select and combine the most appropriate listed fuels for her case of study. Rothermels fire line model is a widely used wildfire spread model, and has been implemented in several simulation tools~\cite{Finney:FARSITE,Bova:IJWF2015,mcgratta_2013} \textbf{OTHERS?}, enabling a range of planning and operational forecasting capabilities. 

% Level set as a simplified fire spread model, operational forecasting.
Evolution of wild land fires is usually tied to a fire front, which can have a complex shape moving/deforming with the local flame spread velocities. Knowing the location of this fire front and local properties of the fire, fuel and terrain allows, by use of simplified models like Rothermels, to predict local ignition and fire spread. When modeling the problem numerically, identifying such fire front at a given moment in time requires a mathematical representation and an associated discretization or mesh. The Lagrangian treatment of the motion of a fire front requires specifying a set of control markers and connectivity that define and move with such front. Therefore, the fire front is represented by this discrete mesh. For example, if the domain of motion is a surface terrain, this Lagrangian mesh can be a polygon that moves with the front along this surface~\cite{Finney:FARSITE,Bova:IJWF2015}. 
On the other hand, an Eulerian representation of motion allows for the mesh to be fixed in space (i.e. in a terrain a polygonal patch mesh of said terrain). Then, the location of the fire front can be portrayed implicitly by using a scalar field defined in the fixed two dimensional mesh. This scalar field is called level set function and numerical methods that evolve it in time are called Level Set Methods (LSM)~\cite{Sethian:1999,Osher:2006}. Some wildfire solvers that use LSM to track the fire front are described in references~\cite{coen_2013,Bova:IJWF2015,mcgratta_2013,LAUTENBERGER_2013}. A practical comparison of the two techniques as implemented in WFDS~\cite{Mell:IJWF2007} and FARSITE~\cite{Finney:FARSITE} is provided in reference~\cite{Bova:IJWF2015}.

% Resolving the flow features on terrains using LES over complex terrains. Describe what we do, FDS, cut-cell schemes, immersed boundary methods.
As the rate of spread of the front line is dependent in the local velocity on empirical fire spread models like Rothermels, a better local resolution of the velocity field around complex terrains should be beneficial to the accuracy of the overall model. In reality, besides the obstructive character of the terrain itself, compounded effects due to the fire affect the local velocity field. \textbf{Some of these are ...}.
The de facto physics-based technique for practical simulation of outdoor flows is Large Eddy Simulation (LES)~\cite{coen_2013,mcgratta_2013}. Within FDS~\cite{mcgratta_2013},  the filtered Navier-Stokes equations for buoyant, combusting flows are evolved using a Low Mach approximation. Chemical species are handled using scalar transport equations. Energy transport is accounted for by means of the solution of radiative transport equations and a thermodynamic divergence proxy. \textbf{Comment on WFDS for LOF. The work of Mell et al.~\cite{Mell:IJWF2007}.}.

% Unstructured geometries, capability can be tuned for use with LOF and fire spread models.
% What has been done, cut-cell scheme IBM, connection to LS and Rothermels, wind.
Lately, substantial work has been done to add to FDS the option of simulating flows around unstructured geometries. These are represented by surface triangulations which is a typical representation in computer aided design (CAD) and simulation software. This added flexibility is still compliant with the structured cartesian grids and differentiation schemes within FDS. We implemented a cut-cell scheme~\cite{berger_2016} to advance the scalar transport and energy of multicomponent mixtures on the polyhedra next to the geometries. These polyhedrons arise from geometry intersection with the cartesian fluid grid. As geometry boundaries are not changed, the method has the property of exact volume and area conservation. Velocities next to geometries are reconstructed using an immersed boundary method (IBM)~\cite{fadlun_2000,balaras_2004} coupled with an equilibrium wall model for turbulent flows~\cite{mcdermo_2018}. Variables related to IBM forcing and heat transfer modeling at the wall are sampled through interpolation and a normal probe approach~\cite{balaras_2004}. Also, radiation transport boundary conditions have been adapted to this technique. 
For fire simulation over complex terrains a range of options in terms of physics fidelity of the model are being developed, in particular, in the representation of fire spread. For fast simulation of outdoor fires, this new capability has been interfaced with a Level Set scheme~\cite{Bova:IJWF2015} and Rothermels fire spread model. The wind speed input for Rothermels model results from the local velocity over the terrain which is affected by fire buoyancy and flow evolution. These are in turn also influenced by wind atmospheric conditions, accounted for within simulations using a mean wind forcing concept.  

% Outline of the manuscript.
In the following section a description on the mathematical model for fire over terrain is provided. The terrain description and model discretization is provided in section~\ref{sec:terraindisc}. An account of the fire spread model is given in section~\ref{sec:firespread}, as well as the wind atmospheric boundary conditions in section~\ref{sec:wind}. Numerical experiments testing the methodology on a simple example and a real case of fire in Cogoleto, Italy are provided in section~\ref{sec:numexp}. Finally a summary of this work is outlined in section~\ref{sec:summary}.


\section{Mathematical model} \label{sec:matmodel}

% Model equations, fluid side : scalar transport, momentum, energy. Stratification.




% Combustion, fuel, fire spread model equations.


\section{Terrain description and discretization} \label{sec:terraindisc}


% How do we describe terrains - geometries. Methodology to get from DEM a geometry.


% What FDS does for Scalar Transport - 1 paragraph Modification close to the terrain surface: scalar transport, momentum, divergence.


% Connection with level set fire spread.



\section{Wildland Fire Spread} \label{sec:firespread}

There are three ways of simulating wildland fire spread.

\begin{enumerate}
\item {\bf Particle Method:} The vegetation is represented by a collection of Lagrangian particles that are heated via convection and radiation.
\item {\bf Boundary Fuel Model:} Ground vegetation is modeled like a porous solid with a thickness equal to the height of the vegetation.
\item {\bf Level Set Method:} The fire front propagates using purely empirical rules.
\end{enumerate}
The Particle Method and Boundary Fuel Model require thermo-physical properties of the vegetative fuels and the fire spread rate is {\em predicted} by the model. The Level Set Method relies on a set of experimentally-determined spread rates for different types of vegetation and wind speeds.


% Here, methodology to get fuels in the terrain from online sources.




% Defining atmospheric conditions:
\section{Defining atmospheric wind : The Mean Forcing Concept} \label{sec:wind}

The process of steering the solution of the mass, momentum, and energy equations to match wind speed and direction gathered at random weather stations is known as \emph{data assimilation}~\cite{Kalnay:2003}. FDS uses a relatively simple data assimilation technique, a method known as \emph{nudging}, where you add a forcing term to the momentum equation to ``nudge'' the flow in the direction of the specified wind. FDS automatically drives the mean velocity components toward the desired values by adding a forcing term to the momentum equation. For example, the $u$ component equation is modified as follows:
\begin{equation}
   \frac{\partial u}{\partial t} + \ldots = \frac{u_0(z,t)-\overline{u}(z,t)}{\tau} \label{mean_forcing_u}
\end{equation}
Here, $u(\mathbf{x},t)$ is the computed velocity component, $u_0$ is the specified wind field component that can vary with height and time, and $\overline{u}$ is the average of the computed velocity component at the height $z$. The relaxation time scale, $\tau$, is an important parameter. The shorter this time scale, the faster the flow field will ``catch up'' to the mean wind, but at the expense of possibly washing out important flow structures. There is no firm guidance in the atmospheric modeling literature.


\section{Numerical Experiments} \label{sec:numexp}





\subsection{Simple example}  \label{sec:simexp}




\subsection{Cogoleto Fire}  \label{sec:cogo}




\section{Summary} \label{sec:summary}







\reftitle{References}


\externalbibliography{yes}
\bibliography{Vanella_Article_Atmosphere_2020}


% The following MDPI journals use author-date citation: Arts, Econometrics, Economies, Genealogy, Humanities, IJFS, JRFM, Laws, Religions, Risks, Social Sciences. For those journals, please follow the formatting guidelines on http://www.mdpi.com/authors/references
% To cite two works by the same author: \citeauthor{ref-journal-1a} (\citeyear{ref-journal-1a}, \citeyear{ref-journal-1b}). This produces: Whittaker (1967, 1975)
% To cite two works by the same author with specific pages: \citeauthor{ref-journal-3a} (\citeyear{ref-journal-3a}, p. 328; \citeyear{ref-journal-3b}, p.475). This produces: Wong (1999, p. 328; 2000, p. 475)




%% for journal Sci
%\reviewreports{\\
%Reviewer 1 comments and authors’ response\\
%Reviewer 2 comments and authors’ response\\
%Reviewer 3 comments and authors’ response
%}

%%%%%%%%%%%%%%%%%%%%%%%%%%%%%%%%%%%%%%%%%%
\end{document}

